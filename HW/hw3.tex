\documentclass[12pt]{article}

\usepackage{times}
\usepackage{amsmath}
\usepackage{hyperref}


% revise margins
\setlength{\headheight}{0.0in}
\setlength{\topmargin}{-0.45in}
\setlength{\headsep}{0.0in}
\setlength{\textheight}{9.55in}
\setlength{\footskip}{0.35in}
\setlength{\oddsidemargin}{-0.25in}
\setlength{\evensidemargin}{-0.25in}
\setlength{\textwidth}{7.0in}

\setlength{\parskip}{6pt}
\setlength{\parindent}{0pt}

\hypersetup{pdfpagemode=UseNone} % don't show bookmarks on initial view
\hypersetup{colorlinks, urlcolor={blue}}

\begin{document}
\thispagestyle{empty}


Stat 877: Statistical methods for molecular biology (Spring, 2019)\\
\textbf{Homework \#3}: QTL mapping

\textbf{Due 14 March 2019}

\bigskip

We will consider a set of simulated data from a backcross with 300 individuals, with a
single quantitative phenotype.  A selective genotyping strategy was
used: only the top 46 and bottom 46 individuals, by phenotype, were
genotyped.

\begin{enumerate}
\item Grab the comma-delimited data file at

{\footnotesize \tt
\verb| | \href{http://www.biostat.wisc.edu/~kbroman/teaching/uwstatgen/hw3.csv}{http://www.biostat.wisc.edu/{\textasciitilde}kbroman/teaching/uwstatgen/hw3.csv}
}

and place it in your R working directory.

Within R, you'll need to install R/qtl via
{\footnotesize \verb|install.packages("qtl")|}

Then load R/qtl via {\footnotesize \verb|library(qtl)|}

Then import the data file via {\footnotesize \verb|hw <- read.cross("csv", file="hw3.csv")|}


\item Use each of standard interval mapping (by the EM algorithm) and
  Haley-Knott regression to map QTL in this cross.  Also, use a
  permutation test to establish significance of identified QTL, and
  calculate 1.5-LOD support intervals for the locations of inferred
  QTL.

Do two versions of the permutation test: the usual kind plus a
stratified permutation test (with individuals stratified by the amount
of genotyping, and with permutations performed within these two strata).

\bigskip
{\footnotesize
To perform the stratified permutation test, do something like this:
\vspace{-12pt}
{\footnotesize
\begin{verbatim}
    nt <- ntyped(hw)
    strat <- as.numeric(nt > mean(unique(nt)))
    operm <- scanone(hw, method="hk", n.perm=1000, perm.strat=strat)
\end{verbatim}
}}

\item How do the results change if you omit the individuals that were
  not genotyped?

\bigskip
{\footnotesize
To drop the non-genotyped, individuals, use code like
\vspace{-12pt}
{\footnotesize
\begin{verbatim}
    hw_sub <- subset(hw, ind=(ntyped(hw) > 0))
\end{verbatim}
}}

\item What do you conclude, regarding the behavior of standard
  interval mapping vs Haley-Knott regression in the presence of
  selective genotyping, and regarding the use of an unstratified vs
  stratified permutation test?

\end{enumerate}




\end{document}
